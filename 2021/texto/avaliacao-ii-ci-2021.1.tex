\documentclass[12pt,a4paper]{article}
\usepackage[utf8]{inputenc}
\usepackage[brazil]{babel}
\usepackage{graphicx}
\usepackage{amssymb, amsfonts, amsmath}
\usepackage{float}
\usepackage{enumerate}
\usepackage[top=1.5cm, bottom=1.5cm, left=1.25cm, right=1.25cm]{geometry}

\DeclareMathOperator{\sen}{sen}

\begin{document}
\pagestyle{empty}

\begin{center}
  \begin{tabular}{ccc}
    \begin{tabular}{c}
      \includegraphics[scale=0.25]{../../biblioteca/imagem/brasao-de-armas-brasil} \\
    \end{tabular} & 
    \begin{tabular}{c}
      Ministério da Educação \\
      Universidade Federal dos Vales do Jequitinhonha e Mucuri \\
      Faculdade de Ciências Sociais, Aplicadas e Exatas - FACSAE \\
      Departamento de Ciências Exatas - DCEX \\
      Disciplina: Cálculo Diferencial e Integral I. \quad Semestre: 2021/1\\
      Prof. Me. Luiz C. M. de Aquino\\
    \end{tabular} &
    \begin{tabular}{c}
      \includegraphics[scale=0.25]{../../biblioteca/imagem/logo-ufvjm} \\
    \end{tabular}
  \end{tabular}
\end{center}

\begin{center}
 \textbf{Avaliação II}
\end{center}

\textbf{Instruções}
\begin{itemize}
 \item Todas as justificativas necessárias na solução de cada questão devem estar presentes nesta avaliação;
 \item As respostas finais de cada questão devem estar escritas de caneta;
 \item Esta avaliação tem um total de 35,0 pontos.
\end{itemize}

\begin{enumerate}
  \item \textbf{[7,0 pontos]} Calcule a derivada de $f$ usando a definição por limite.
    \begin{enumerate}
     \item $f(x) = x^2 - 3x + 4$
     \item $f(x) = \sqrt{x^2 + 1}$
    \end{enumerate}

  \item \textbf{[7,0 pontos]} Calcule a derivada de $f$.
    \begin{enumerate}
     \item $f(x) = \left(4x^3 - 3x^2\right)^2$
     \item $f(x) = \dfrac{\cos\sqrt{x}}{x^2}$
    \end{enumerate}

  \item \textbf{[7,0 pontos]} Calcule a equação da reta tangente ao gráfico de $f$
    no ponto $x_0$ dado.
    \begin{enumerate}
     \item $f(x) = \dfrac{1}{x}$, $x_0 = \dfrac{1}{2}$
     \item $f(x) = x^2\sen x$, $x_0 = \pi$
    \end{enumerate}
     
  \item \textbf{[7,0 pontos]} Encontre as retas tangentes à parábola 
    $y = x^2 - 4x$ que passam pelo ponto $(5,\, 1)$.

  \item \textbf{[7,0 pontos]} Suponha que os gráficos das funções
    $f(x) = x^2 - 2x - 1$ e $g(x) = ax^2 + bx + c$ possuem a mesma reta tangente
    em $x_0 = -1$. Determine a expressão da função $g$ sabendo que seu gráfico
    passa pelo ponto $(-3,\, 6)$.
  
  \end{enumerate}
\end{document}
