\documentclass[12pt,a4paper]{article}
\usepackage[utf8]{inputenc}
\usepackage[brazil]{babel}
\usepackage{graphicx}
\usepackage{amssymb, amsfonts, amsmath}
\usepackage{float}
\usepackage{enumerate}
\usepackage[top=1.5cm, bottom=1.5cm, left=1.25cm, right=1.25cm]{geometry}

\begin{document}
\pagestyle{empty}

\begin{center}
  \begin{tabular}{ccc}
    \begin{tabular}{c}
      \includegraphics[scale=0.25]{../../biblioteca/imagem/brasao-de-armas-brasil} \\
    \end{tabular} & 
    \begin{tabular}{c}
      Ministério da Educação \\
      Universidade Federal dos Vales do Jequitinhonha e Mucuri \\
      Faculdade de Ciências Sociais, Aplicadas e Exatas - FACSAE \\
      Departamento de Ciências Exatas - DCEX \\
      Disciplina: Cálculo Diferencial e Integral I. \quad Semestre: 2021/1\\
      Prof. Me. Luiz C. M. de Aquino\\
    \end{tabular} &
    \begin{tabular}{c}
      \includegraphics[scale=0.25]{../../biblioteca/imagem/logo-ufvjm} \\
    \end{tabular}
  \end{tabular}
\end{center}

\begin{center}
 \textbf{Avaliação I}
\end{center}

\textbf{Instruções}
\begin{itemize}
 \item Todas as justificativas necessárias na solução de cada questão devem estar presentes nesta avaliação;
 \item As respostas finais de cada questão devem estar escritas de caneta;
 \item Esta avaliação tem um total de 30,0 pontos.
\end{itemize}

\begin{enumerate}
  \item \textbf{[8,0 pontos]} Calcule os limites abaixo.
    \begin{enumerate}
     \item $\displaystyle \lim_{x\to 4} \dfrac{x^2 - 16}{x - 4}$
     \item $\displaystyle \lim_{x\to 1} \dfrac{\sqrt{x^2 - x + 1} - \sqrt{x}}{x - 1}$
     \item $\displaystyle \lim_{h\to 0} \dfrac{\dfrac{1}{x + h} - \dfrac{1}{x}}{h}$
    \end{enumerate}

  \item \textbf{[8,0 pontos]} Calcule os seguintes limites infinitos.
    \begin{enumerate}
     \item $\displaystyle \lim_{x\to 2^+} \dfrac{x - 1}{4 - x^2}$
     \item $\displaystyle \lim_{x\to 4^+} \dfrac{x^2 - 3x - 5}{(x - 4)^3}$
     \item $\displaystyle \lim_{x\to 1^-} \dfrac{1}{x(x - 1)(x - 2)}$
    \end{enumerate}

  \item \textbf{[8,0 pontos]} Calcule os seguintes limites no infinito.
    \begin{enumerate}
     \item $\displaystyle \lim_{x\to +\infty} \dfrac{3x^2 + 2}{2x^2 + 3}$
     \item $\displaystyle \lim_{x\to +\infty} \dfrac{x}{\sqrt{2x^2 + 1}}$
     \item $\displaystyle \lim_{x\to +\infty} x - \sqrt{x^2 -6x + 1}$
    \end{enumerate}
 
  \item \textbf{[6,0 pontos]} Encontre um exemplo no qual existe 
  $\displaystyle \lim_{x\to 2} f(x) + g(x)$, mas não existem 
  $\displaystyle \lim_{x\to 2} f(x)$ e 
  $\displaystyle \lim_{x\to 2} g(x)$.
    
  \end{enumerate}
\end{document}
