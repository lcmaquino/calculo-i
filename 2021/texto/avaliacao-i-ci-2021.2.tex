\documentclass[12pt,a4paper]{article}
\usepackage[utf8]{inputenc}
\usepackage[brazil]{babel}
\usepackage{graphicx}
\usepackage{amssymb, amsfonts, amsmath}
\usepackage{float}
\usepackage{enumerate}
\usepackage[top=1.5cm, bottom=1.5cm, left=1.25cm, right=1.25cm]{geometry}

\begin{document}
\pagestyle{empty}

\begin{center}
  \begin{tabular}{ccc}
    \begin{tabular}{c}
      \includegraphics[scale=0.25]{../../biblioteca/imagem/brasao-de-armas-brasil} \\
    \end{tabular} & 
    \begin{tabular}{c}
      Ministério da Educação \\
      Universidade Federal dos Vales do Jequitinhonha e Mucuri \\
      Faculdade de Ciências Sociais, Aplicadas e Exatas - FACSAE \\
      Departamento de Ciências Exatas - DCEX \\
      Disciplina: Cálculo Diferencial e Integral I. \quad Semestre: 2021/2\\
      Prof. Me. Luiz C. M. de Aquino\\
      Discente: \rule{6cm}{0.01cm} Data: \rule{0.5cm}{0.01cm}/\rule{0.5cm}{0.01cm}/\rule{1cm}{0.01cm}\\
    \end{tabular} &
    \begin{tabular}{c}
      \includegraphics[scale=0.25]{../../biblioteca/imagem/logo-ufvjm} \\
    \end{tabular}
  \end{tabular}
\end{center}

\begin{center}
 \textbf{Avaliação I}
\end{center}

\textbf{Instruções}
\begin{itemize}
 \item Todas as justificativas necessárias na solução de cada questão devem estar presentes nesta avaliação;
 \item As respostas finais de cada questão devem estar escritas de caneta;
 \item Esta avaliação tem um total de 30,0 pontos.
\end{itemize}

\begin{enumerate}
  \item \textbf{[9,0 pontos]} Considerando a função 
    $f(x)=\begin{cases}
      2x+16 & ;\, x<-3\\
      x^{2}-x-2 & ;\,-3\leq x<3\\
      \dfrac{10}{3}x-5 & ;\, x\geq3
    \end{cases}$, calcule os limites abaixo.

    \begin{tabular}{lll}
      (a) ${\displaystyle \lim_{x\to-3^{-}}f(x)}$ & 
      (c) ${\displaystyle \lim_{x\to 1^{-}}f(x)}$ & 
      (e) ${\displaystyle \lim_{x\to3^{-}}f(x)}$ \\
      (b) ${\displaystyle \lim_{x\to-3^{+}}f(x)}$ &
      (d) ${\displaystyle \lim_{x\to 1^{+}}f(x)}$ & 
      (f) ${\displaystyle \lim_{x\to3^{+}}f(x)}$
    \end{tabular}

  \item \textbf{[9,0 pontos]} Calcule os seguintes.

    \begin{tabular}{lll}
      (a) ${\displaystyle \lim_{x\to2}\dfrac{x^{2}-x-2}{x-2}}$ & 
      (c) ${\displaystyle \lim_{y\to9}\dfrac{9-y}{3-\sqrt{y}}}$ & 
      (e) ${\displaystyle \lim_{t\to1}\dfrac{t^{2}+t-2}{t^{2}-3t+2}}$ \\
      (b) ${\displaystyle \lim_{h\to0}\dfrac{(h-5)^{2}-25}{h}}$ & 
      (d) ${\displaystyle \lim_{x\to1}\dfrac{\sqrt{x}-x}{1-\sqrt{x}}}$ &  
      (f) ${\displaystyle \lim_{m \to 1}\dfrac{m^3 - 1}{\sqrt{m} - 1}}$
    \end{tabular}

  \item \textbf{[6,0 pontos]} Em um estacionamento é cobrado R\$ 2,00 por cada intervalo de 60
    minutos (ou partes do mesmo). Com base nessa informação, responda
    aos quesitos abaixo.

    \begin{enumerate}
      \item Qual é o valor pago por 60, 100 e 110 minutos?
      \item Seja $f$ a função que associa a quantidade de minutos de permanência
        no estacionamento com o valor pago pelo serviço. O limite ${\displaystyle \lim_{x\to 30} f(x)}$
        existe? E quanto a ${\displaystyle \lim_{x\to 60}f(x)}$? Justifique sua resposta.
      \item Esboce o gráfico da função $f$ do quesito anterior.
    \end{enumerate}

 
  \item \textbf{[6,0 pontos]} Escolha um número positivo não nulo qualquer. 
   Utilizando uma calculadora, calcule a sua raiz quadrada. Em seguida, calcule 
   a raiz quadrada do resultado anterior. Continuando esse processo por várias vezes, 
   o resultado fica cada vez mais próximo de 1. Use os conceitos de limite para justificar esse fato.

  \end{enumerate}
\end{document}
