\documentclass[12pt,a4paper]{article}
\usepackage[utf8]{inputenc}
\usepackage[brazil]{babel}
\usepackage{graphicx}
\usepackage{amssymb, amsfonts, amsmath}
\usepackage{float}
\usepackage{enumerate}
\usepackage[top=1.5cm, bottom=1.5cm, left=1.25cm, right=1.25cm]{geometry}

\DeclareMathOperator{\sen}{sen}
\DeclareMathOperator{\tg}{tg}
\DeclareMathOperator{\arcsen}{arcsen}
\DeclareMathOperator{\arctg}{arctg}

\begin{document}
\pagestyle{empty}

\begin{center}
  \begin{tabular}{ccc}
    \begin{tabular}{c}
      \includegraphics[scale=0.25]{../../biblioteca/imagem/brasao-de-armas-brasil} \\
    \end{tabular} & 
    \begin{tabular}{c}
      Ministério da Educação \\
      Universidade Federal dos Vales do Jequitinhonha e Mucuri \\
      Faculdade de Ciências Sociais, Aplicadas e Exatas - FACSAE \\
      Departamento de Ciências Exatas - DCEX \\
      Disciplina: Cálculo Diferencial e Integral I. \quad Semestre: 2021/2\\
      Prof. Me. Luiz C. M. de Aquino\\
      Discente: \rule{6cm}{0.01cm} Data: \rule{0.5cm}{0.01cm}/\rule{0.5cm}{0.01cm}/\rule{1cm}{0.01cm}\\
    \end{tabular} &
    \begin{tabular}{c}
      \includegraphics[scale=0.25]{../../biblioteca/imagem/logo-ufvjm} \\
    \end{tabular}
  \end{tabular}
\end{center}

\begin{center}
 \textbf{Avaliação II}
\end{center}

\textbf{Instruções}
\begin{itemize}
 \item Todas as justificativas necessárias na solução de cada questão devem estar presentes nesta avaliação;
 \item As respostas finais de cada questão devem estar escritas de caneta;
 \item Esta avaliação tem um total de 35,0 pontos.
\end{itemize}

\begin{enumerate}
 \item \textbf{[7,0 pontos]} Calcule a derivada das funções definidas abaixo.

 \begin{enumerate}
  \item $f(r) = \dfrac{e^r - r^2}{r^3 + r}$.
  \item $g(u) = \cos^2 u - \sen^2 u$.
 \end{enumerate}

 \item \textbf{[7,0 pontos]} Determine a reta tangente ao gráfico das funções definidas abaixo nos pontos indicados.

 \begin{enumerate}
  \item $f(x) = \dfrac{x - 1}{x + 1}$, $P=\left(3;\, \dfrac{1}{2}\right)$.
  \item $j(x) = \cos x\,e^x$, $P=\left(0;\, 1\right)$.
 \end{enumerate}
   
  \item \textbf{[7,0 pontos]} Usando o fato de que $[\cos x]^\prime = -\sen x$, exiba um desenvolvimento para justificar que 
$[\arccos x]^\prime = -\dfrac{1}{\sqrt{1 - x^2}}$.

  \item \textbf{[7,0 pontos]} Usando o fato de que $[\tg x]^\prime = \sec^2 x$, exiba um desenvolvimento para justificar que 
$[\arctg x]^\prime = \dfrac{1}{1 + x^2}$.
 
  \item \textbf{[7,0 pontos]} Determine o ponto de interseção entre o eixo $x$ e a reta tangente ao gráfico de 
  $f(x)=\dfrac{1}{4}x^{2} -3x + 10$ no ponto $(2,\, f(2))$.

  \end{enumerate}
\end{document}
