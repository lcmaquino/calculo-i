\documentclass[12pt,a4paper]{article}
\usepackage[utf8]{inputenc}
\usepackage[brazil]{babel}
\usepackage{graphicx}
\usepackage{amssymb, amsfonts, amsmath}
\usepackage{float}
\usepackage{enumerate}
\usepackage[top=1.5cm, bottom=1.5cm, left=1.25cm, right=1.25cm]{geometry}

\DeclareMathOperator{\sen}{sen}

\begin{document}
\pagestyle{empty}

\begin{center}
  \begin{tabular}{ccc}
    \begin{tabular}{c}
      \includegraphics[scale=0.25]{../../biblioteca/imagem/brasao-de-armas-brasil} \\
    \end{tabular} & 
    \begin{tabular}{c}
      Ministério da Educação \\
      Universidade Federal dos Vales do Jequitinhonha e Mucuri \\
      Faculdade de Ciências Sociais, Aplicadas e Exatas - FACSAE \\
      Departamento de Ciências Exatas - DCEX \\
      Disciplina: Cálculo Diferencial e Integral I. \quad Semestre: 2021/2\\
      Prof. Me. Luiz C. M. de Aquino\\
      Discente: \rule{6cm}{0.01cm} Data: \rule{0.5cm}{0.01cm}/\rule{0.5cm}{0.01cm}/\rule{1cm}{0.01cm}\\
    \end{tabular} &
    \begin{tabular}{c}
      \includegraphics[scale=0.25]{../../biblioteca/imagem/logo-ufvjm} \\
    \end{tabular}
  \end{tabular}
\end{center}

\begin{center}
 \textbf{Avaliação III}
\end{center}

\textbf{Instruções}
\begin{itemize}
 \item Todas as justificativas necessárias na solução de cada questão devem estar presentes nesta avaliação;
 \item As respostas finais de cada questão devem estar escritas de caneta;
 \item Esta avaliação tem um total de 35,0 pontos.
\end{itemize}

\begin{enumerate}
  \item Considere a função dada por $f(x) = 3x^3 - 4x$. 
    Determine o que for solicitado abaixo.
  
    \begin{enumerate}
      \item \textbf{[2,0 pontos]} Determine os intervalos de crescimento ou de decrescimento de $f$.
      \item \textbf{[2,0 pontos]} Determine os intervalos nos quais a concavidade do gráfico de $f$ é para baixo
        ou para cima.
      \item \textbf{[4,0 pontos]} Esboce o gráfico de f.
    \end{enumerate}

  \item \textbf{[9,0 pontos]} Determine a constante $c$ tal que o gráfico da função dada 
  por $f(x) = \left(1 -\dfrac{2c}{3}\right)x^3 + (3 - 2c)x$ seja sempre decrescente.
   
  \item \textbf{[9,0 pontos]} Ache a maior área possível do triângulo formado no primeiro 
  quadrante pelos eixos $x$ e $y$ e pela reta passando pelo ponto $(2,\, 5)$.

  \item \textbf{[9,0 pontos]} Determine o ponto da parábola $y = 2x^2$ que está mais próximo do ponto $(0,\, 1)$. 
    
  \end{enumerate}
\end{document}
